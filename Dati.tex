%----------------------------------------------------------------------------------------
%   USEFUL COMMANDS
%----------------------------------------------------------------------------------------

\newcommand{\dipartimento}{Dipartimento di Matematica ``Tullio Levi-Civita''}

%----------------------------------------------------------------------------------------
% 	USER DATA
%----------------------------------------------------------------------------------------

% Data di approvazione del piano da parte del tutor interno
% compilare inserendo al posto di GG 2 cifre per il giorno, e al posto di 
% AAAA 4 cifre per l'anno
\newcommand{\dataApprovazione}{GG Mese AAAA}

% Dati dello Studente
\newcommand{\nomeStudente}{Federico}
\newcommand{\cognomeStudente}{Vegro}
\newcommand{\matricolaStudente}{1009448}
\newcommand{\emailStudente}{federico.vegro@studenti.unipd.it}
\newcommand{\telStudente}{+ 39 340 5674967}

% Dati del Tutor Aziendale
\newcommand{\nomeTutorAziendale}{Paolo}
\newcommand{\cognomeTutorAziendale}{Moro}
\newcommand{\emailTutorAziendale}{info@fablabnetwork.it}
\newcommand{\telTutorAziendale}{+ 39 320 0644621}
\newcommand{\ruoloTutorAziendale}{Project manager}

% Dati dell'Azienda
\newcommand{\ragioneSocAzienda}{Lab Network S.r.l.}
\newcommand{\indirizzoAzienda}{Viale Della Navigazione Interna, 51/A - 35129 Padova (PD)}
\newcommand{\sitoAzienda}{https://www.fablabnetwork.it}

% Dati del Tutor Interno (Docente)
\newcommand{\titoloTutorInterno}{Prof.}
\newcommand{\nomeTutorInterno}{Tullio}
\newcommand{\cognomeTutorInterno}{Vardanega}

\newcommand{\prospettoSettimanale}{
     % Personalizzare indicando in lista, i vari task settimana per settimana
    \begin{itemize}
        \item \textbf{Prima Settimana}
        \begin{itemize}
        		\item Incontro con gli stakeholders al fine di definire i requisiti e le modalità di 
        		sviluppo del sistema richiesto;
            \item Verifica strumenti di lavoro assegnati;
        \end{itemize}
        \item \textbf{Seconda Settimana - Formazione} 
        \begin{itemize}
            \item Studio preliminare sull'infrastruttura esistente;
        \end{itemize}
        \item \textbf{Terza Settimana - Interazione Hardware Software} 
        \begin{itemize}
            \item Adattamento del software su Arduino per il nuovo modulo;
        \end{itemize}
        \item \textbf{Quarta Settimana - Software} 
        \begin{itemize}
            \item Adattamento del software crm esistente per l'aggiunta del nuovo modulo;
        \end{itemize}
        \item \textbf{Quinta Settimana - Modellazione 3D} 
        \begin{itemize}
            \item Creazione dell'involucro esterno per il modulo utente;
            \item Stampa 3D del modello e suo collaudo;
        \end{itemize}
        \item \textbf{Sesta e Settima Settimana - Piattaforma online - server side} 
        \begin{itemize}
            \item Pianificazione dello sviluppo della piattaforma online per la parte pubblica (sito);
            \item Sviluppo della piattaforma con iniziale attenzione al lato server;
        \end{itemize}
        \item \textbf{Ottava e Nona Settimana - Interfaccia grafica} 
        \begin{itemize}
            \item Cura dell'interfaccia grafica della piattaforma online;
            \item Ottimizzazione dell'interfaccia grafica con particolare attenzione all'uso multi-piattaforma;
            \item Test e verifiche finali;
        \end{itemize}
        \item \textbf{Decima Settimana - Test} 
        \begin{itemize}
            \item Test e verifica dei bug e finalizzazione;
        \end{itemize}
        \item \textbf{Undicesima Settimana - Redazione manuali} 
        \begin{itemize}
            \item Stesura dei manuali d'uso per l'utente e l'amministratore;
        \end{itemize}
        \item \textbf{Dodicesima Settimana - Collaudo} 
        \begin{itemize}
            \item Collaudo del sistema completo;
        \end{itemize}
        \item \textbf{Tredicesima Settimana - Presentazione finale} 
        \begin{itemize}
            \item Presentazione del prodotto finito in presenza degli stakeholders;
            \item Redazione finale.
        \end{itemize}
    \end{itemize}
}

% Indicare il totale complessivo (deve essere compreso tra le 300 e le 320 ore)
\newcommand{\totaleOre}{308}

\newcommand{\obiettiviObbligatori}{
	 \item \underline{\textit{O01}}: primo obiettivo;
	 \item \underline{\textit{O02}}: secondo obiettivo;
	 \item \underline{\textit{O03}}: terzo obiettivo;
	 
}

\newcommand{\obiettiviDesiderabili}{
	 \item \underline{\textit{D01}}: primo obiettivo;
	 \item \underline{\textit{D02}}: secondo obiettivo;
}

\newcommand{\obiettiviFacoltativi}{
	 \item \underline{\textit{F01}}: primo obiettivo;
	 \item \underline{\textit{F02}}: secondo obiettivo;
	 \item \underline{\textit{F03}}: terzo obiettivo;
}